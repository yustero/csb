\documentclass{article}

\usepackage[T1]{fontenc}
\usepackage{graphicx}
\usepackage{pict2e}
\usepackage{xcolor}
\usepackage{amsmath}
\usepackage[rflt]{floatflt}
\usepackage{graphicx,subfigure,epic,eepic}
\usepackage[most]{tcolorbox}
\usepackage{float}
\usepackage{caption}
\usepackage{fullpage}
\usepackage{hyperref}
\usepackage{fancyhdr}
\pagestyle{fancy}
\lhead{}
\rhead{}
\cfoot{}
\usepackage[top=5mm,includehead,headheight=45pt,
             left=1.5cm,bottom=2cm,right=1.5cm,headsep=0.3cm]{geometry} 

%inkscapestuff
\usepackage{import}
\usepackage{pdfpages}
\usepackage{transparent}
\usepackage{xcolor}

\newcommand{\incfig}[2][1]{%
    \def\svgwidth{#1\columnwidth}
    \import{./figures/}{#2.pdf_tex}
}

\pdfsuppresswarningpagegroup=1





\title{Racipe}
\author{Va}

\begin{document}
\maketitle
Refrence: Original paper by racipe

How do we settle on this regulation function, how different behaviour can be with a different fitness function and how well do boolean simulation capture all the functions with the same positive/negative trends

What are agent based models?

Apparently, to understand GRNs even reaction diffusion models have been used damn 

RACIPE satisfies something called half function rule for each link I dont knwo what exactly it means
 The term I was looking for was "statistical learnin gmethods"

 What is hierarchical cluster analysis?

 Why do we use fold change and what is the role of it? does it somehow sample through positive and negative feedback dimension? Hmm what is it? Why don't we use two independent variables a and b to deal with fold change


\end{document}

