\documentclass{article}

\usepackage[T1]{fontenc}
\usepackage{graphicx}
\usepackage{pict2e}
\usepackage{xcolor}
\usepackage{amsmath}
\usepackage[rflt]{floatflt}
\usepackage{graphicx,subfigure,epic,eepic}
\usepackage[most]{tcolorbox}
\usepackage{float}
\usepackage{caption}
\usepackage{fullpage}
\usepackage{hyperref}
\usepackage{fancyhdr}
\pagestyle{fancy}
\lhead{}
\rhead{}
\cfoot{}
\usepackage[top=5mm,includehead,headheight=45pt,
             left=1.5cm,bottom=2cm,right=1.5cm,headsep=0.3cm]{geometry} 

%inkscapestuff
\usepackage{import}
\usepackage{pdfpages}
\usepackage{transparent}
\usepackage{xcolor}

\newcommand{\incfig}[2][1]{%
    \def\svgwidth{#1\columnwidth}
    \import{./figures/}{#2.pdf_tex}
}

\pdfsuppresswarningpagegroup=1





\title{Equations}
\author{Vaibhav}

\begin{document}
\maketitle

\section{Drive}

It quantifies how much "effective regulation"  a node is under at a certain time step. It is defined as follows for the J'th node 

\[  \frac { | \sum_{ i } ^ { } adj[i][j]*S_i  | } {I_j}                     \]
 
Here, $ I_j $ is the indegree of the given node and $ S_i $ is the state of the node i at a certain time step. 

\[  \frac {  \sum_{ i } ^ { } adj[i][j]*S_i   } {I_j}                     \]
 iohfw
\end{document}
	

