\documentclass{article}

\usepackage[T1]{fontenc}
\usepackage{graphicx}
\usepackage{pict2e}
\usepackage{xcolor}
\usepackage{amsmath}
\usepackage[rflt]{floatflt}
\usepackage{graphicx,subfigure,epic,eepic}
\usepackage[most]{tcolorbox}
\usepackage{float}
\usepackage{caption}
\usepackage{fullpage}
\usepackage{hyperref}
\usepackage{fancyhdr}
\pagestyle{fancy}
\lhead{}
\rhead{}
\cfoot{}
\usepackage[top=5mm,includehead,headheight=45pt,
             left=1.5cm,bottom=2cm,right=1.5cm,headsep=0.3cm]{geometry} 

%inkscapestuff
\usepackage{import}
\usepackage{pdfpages}
\usepackage{transparent}
\usepackage{xcolor}

\newcommand{\incfig}[2][1]{%
    \def\svgwidth{#1\columnwidth}
    \import{./figures/}{#2.pdf_tex}
}

\pdfsuppresswarningpagegroup=1





\title{To Do}
\author{Vaibhav}

\begin{document}
\maketitle

\section{May}
\subsection{8th May}
\begin{itemize}

\item Simulate four networks: One node, one node with self activation, one node with self inhibition and a toggle switch. 

\item Not all networks have a steady state. Find out what it means to be a steady state and which networks have a steady state. 

\end{itemize}

\end{document}

