\documentclass{article}

\usepackage[T1]{fontenc}
\usepackage{graphicx}
\usepackage{pict2e}
\usepackage{xcolor}
\usepackage{amsmath}
\usepackage[rflt]{floatflt}
\usepackage{graphicx,subfigure,epic,eepic}
\usepackage[most]{tcolorbox}
\usepackage{float}
\usepackage{caption}
\usepackage{fullpage}
\usepackage{hyperref}
\usepackage{fancyhdr}
\pagestyle{fancy}
\lhead{}
\rhead{}
\cfoot{}
\usepackage[top=5mm,includehead,headheight=45pt,
             left=1.5cm,bottom=2cm,right=1.5cm,headsep=0.3cm]{geometry} 

%inkscapestuff
\usepackage{import}
\usepackage{pdfpages}
\usepackage{transparent}
\usepackage{xcolor}

\newcommand{\incfig}[2][1]{%
    \def\svgwidth{#1\columnwidth}
    \import{./figures/}{#2.pdf_tex}
}

\pdfsuppresswarningpagegroup=1





\title{Reading: Low dimensionality of phenotypic space as an emergent property
of coordinated teams in biological regulatory networks
}
\author{Vai}

\begin{document}
\maketitle 
\begin{itemize}

\item  In differentiation events, for example, the cells move from a high-entropy
progenitor phenotype to a low-entropy differentiated phenotype. 

What does entropy here mean?
\item Recently,
Tripathi et al. showed that most of the variance in the phenotypic space emergent from the regulatory
networks is explained by the first principal component of the steady state space.

\item As demonstrated
in our previous work, the influence matrix generated from this network (that takes a weighted sum
of direct and indirect interactions between each node pair) contains a higher-order toggle switch. We
refer to each “supernode” (collection of similarly behaving nodes) in this toggle switch as a “team”
consisting of a set of nodes positively influencing each other and negatively influencing the nodes of
the other team (Figure 1A).

\item We then performed principal component analysis over the collection of all steady states obtained
for this network. When the solutions are plotted along the first and second principal component axes,
we found the steady states separated into 3 clusters (Figure 1C, i). Analyzing the expression patterns
of the steady states forming these clusters revealed that the two prominent clusters were the
epithelial and mesenchymal phenotypes. In contrast, the smaller cluster showed a mixed expression
of epithelial (E) and mesenchymal (M) nodes and hence could be classified as the hybrid

\item we hypothesized that the relative weights of node
expressions that make up the PC1 axis could be driven by the two-team structure in the network. To
verify this hypothesis, we looked at the composition of the PC1 axis (using the node coefficients of
PC1) and compared it against the composition of teams (Figure 1D, nodes belonging to different
teams are colored differently). In line with our hypothesis, we see that the PC1 contribution of nodes
belonging to different teams has opposite signs and that belonging to the same team has the same
sign. Together these results support the hypothesis that team structure leads to PC1 explaining most
of the variance in steady-state space.
An immediate implication of the above hypothesis would be that in the absence of strong teams,
the PC1 axis would no longer be able to explain the majority of the variance in the steady-state space.
To test the validity of this implication, we generated random networks by swapping many pairs of
edges in the network (Figure 1E, i)). We have previously shown that such randomization can disrupt
the team structure and thus the stability and frequency of E, M, and H phenotypes (Figure 1E, ii). 

\item Does RACIPE switch nodes? 


\item  A scatter plot
between the team strength of the random and WT networks against the variance explained along the
PC1 axis (Figure 2A, ii) revealed a positive correlation between the two measures, suggesting that an
increase in team strength can lead to a corresponding increase in the variance explained by PC1. An
important observation here is the sigmoidal nature of the data. After a team strength of about 0.25-3,
the variance explained by PC1 starts to saturate.

Does this explain the redundancy? 


\item \textbf{ Question: } The clustering represents bimodal distriubtion caused due to negative regulation of both teams. What kind of distribution would positive regulation induce and here it is shown that there is a very strict bimodal distrbution. Could postive regulation within teams somehow influence the spread of the two modes?

\end{itemize}
\end{document}
	
